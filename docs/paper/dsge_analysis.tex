\documentclass[11pt,a4paper]{article}
\usepackage[utf8]{inputenc}
\usepackage{amsmath, amssymb, amsthm}
\usepackage{graphicx}
\usepackage{booktabs}
\usepackage{hyperref}
\usepackage{geometry}
\geometry{margin=1in}

\title{Analysis of the US Macroeconomic Dynamics: \\ A DSGE Model Perspective}
\author{Antigravity \& Daniel}
\date{\today}

\begin{document}

\maketitle

\begin{abstract}
This paper investigates the dynamic properties of a medium-scale New Keynesian DSGE model (M1002). We perform several experiments, including Impulse Response Function (IRF) analysis for monetary and technology shocks, and sensitivity tests on key structural parameters. Our results highlight the importance of the Taylor rule's aggressiveness and price stickiness in shaping the transmission of shocks to the real economy.
\end{abstract}

\section{Introduction}
Modern macroeconomics relies heavily on Dynamic Stochastic General Equilibrium (DSGE) models for policy analysis and forecasting. This study focuses on the M1002 model, which incorporates several frictions, including price and wage stickiness, habit formation, and financial frictions.

\section{The Model}
The model is a medium-scale New Keynesian framework with several nominal and real rigidities. Key structural parameters are summarized in Table \ref{tab:params}.

\begin{center}
\begin{table}
\caption{Model Parameters}
\label{tab:params}
\begin{tabular}{llll}
\toprule
Name & Symbol & Value & Description \\
\midrule
alp & \alpha & 0.1596 &  \\
zeta_p & \zeta_p & 0.8940 &  \\
iota_p & \iota_p & 0.1865 &  \\
h & h & 0.5347 &  \\
ppsi & \psi & 0.6862 &  \\
psi1 & \psi_1 & 1.3679 &  \\
psi2 & \psi_2 & 0.0388 &  \\
psi3 & \psi_3 & 0.2464 &  \\
rho & \rho_R & 0.7126 &  \\
pi_star & \pi_* & 1.0050 &  \\
\bottomrule
\end{tabular}
\end{table}

\end{center}

\subsection{Economic Agents and Frictions}
The M1002 model is a medium-scale DSGE framework that extends the standard New Keynesian model with several "real" and "nominal" frictions to better match the persistence found in macroeconomic data.

\subsubsection{Households}
Households maximize an intertemporal utility function. Two key features distinguish their behavior:
\begin{itemize}
    \item \textbf{Habit Formation ($h$):} Consumption decisions are persistent. Current utility depends not just on current consumption $c_t$, but on consumption relative to a fraction $h$ of past consumption $c_{t-1}$. This generates the smooth "hump-shaped" response of consumption seen in IRFs.
    \item \textbf{Wage Stickiness ($\zeta_w$):} Households provide differentiated labor services. Following the Calvo (1983) framework, they can only re-optimize their nominal wage with a probability $1-\zeta_w$ each period. This prevents wages from adjusting instantly to labor market conditions, introducing "sticky wages."
\end{itemize}

\subsubsection{Firms and Production}
The production sector is divided into final-goods firms (perfectly competitive) and intermediate-goods firms (monopolistically competitive):
\begin{itemize}
    \item \textbf{Production Function:} Firms use a Cobb-Douglas technology: $y_t = k_t^\alpha (e^{z_t} L_t)^{1-\alpha} - \Phi$, where $\alpha$ is the capital share and $\Phi$ represents fixed costs.
    \item \textbf{Price Stickiness ($\zeta_p$):} Similar to wages, intermediate firms face Calvo pricing frictions. A fraction $\zeta_p$ of firms keep their prices unchanged (or indexed to past inflation $\iota_p$), leading to a slow adjustment of the aggregate price level—the core of the New Keynesian Phillips Curve.
\end{itemize}

\subsubsection{Investment and Capital}
The model incorporates **Investment Adjustment Costs ($S''$)** and **Variable Capital Utilization ($\psi$)**. These features ensure that the shadow price of capital ($q_t$) and the investment rate ($i_t$) do not jump erratically, but rather follow smooth paths consistent with historical investment data.

\subsubsection{Financial Frictions (The BGG Framework)}
A distinguishing feature of the M1002 model relative to simpler versions (like Smets-Wouters) is the inclusion of a **Financial Accelerator** based on Bernanke, Gertler, and Gilchrist (1999).
\begin{itemize}
    \item \textbf{Entrepreneurs:} They borrow from financial intermediaries to purchase capital. However, because of asymmetric information and "costly state verification," they face an external finance premium (spread) over the risk-free rate.
    \item \textbf{The Spread:} The spread depends inversely on the entrepreneur's net worth. In a downturn, asset prices ($q_t$) fall, reducing net worth, which hikes the borrowing cost and further depresses investment—a feedback loop known as the "Financial Accelerator."
\end{itemize}

\subsubsection{The Central Bank}
The monetary authority follows a **Taylor Rule** with interest rate smoothing ($\rho$). It reacts to deviations of inflation from the target ($\pi^*$) and to the "output gap"—the difference between actual output and the level of output that would exist if all prices and wages were flexible ($y_t - y_t^f$). 

\subsection{Data and Identification}
The model is designed to be estimated using live macroeconomic data from the Federal Reserve Economic Data (FRED) database. We utilize 13 key observables:
\begin{enumerate}
    \item \textbf{Output Growth}: Real GDP per capita growth (GDP, CNP16OV).
    \item \textbf{Inflation}: GDP Deflator (GDPDEF) and Core PCE (PCEPILFE).
    \item \textbf{Interest Rates}: Effective Federal Funds Rate (DFF).
    \item \textbf{Labor Market}: Average weekly hours and compensation (AWHNONAG, COMPNFB).
    \item \textbf{Financial Spreads}: BAA-10Y Treasury spread.
\end{enumerate}

Raw data is transformed from levels to log-deviations or growth rates to match the model's stationary states. 

\subsection{Calibration and Methodology}
The model parameters are calibrated or estimated using a Bayesian approach. Parameters like the depreciation rate ($\delta = 0.025$) and the steady-state capital share ($\alpha = 0.3$) are often fixed to standard values, while others like price stickiness ($\zeta_p$) and Taylor rule coefficients ($\psi_i$) are estimated by maximizing the posterior density:
\[ \mathcal{P}(\theta | Y) \propto \mathcal{L}(Y | \theta) \pi(\theta) \]
where $\mathcal{L}(Y | \theta)$ is the likelihood computed via the Kalman Filter and $\pi(\theta)$ represents the prior distributions.

\section{Experiments and Results}

\subsection{Baseline Impulse Responses}
We analyze how the economy reacts to exogenous shocks over a 40-quarter horizon. Figure \ref{fig:irf_monetary} shows the impact of a contractionary monetary policy shock. 

\begin{figure}[htbp]
    \centering
    \includegraphics[width=0.8\textwidth]{../../outputs/plots/irf_rm_sh.png}
    \caption{IRFs - Monetary Policy Shock (40 Quarters)}
    \label{fig:irf_monetary}
\end{figure}

Figure \ref{fig:irf_tech} displays the response to a neutral technology shock. A boost in productivity increases output while putting downward pressure on inflation.

\begin{figure}[htbp]
    \centering
    \includegraphics[width=0.8\textwidth]{../../outputs/plots/irf_ztil_sh.png}
    \caption{IRFs - Technology Shock (40 Quarters)}
    \label{fig:irf_tech}
\end{figure}

\subsection{Sensitivity Analysis: Taylor Rule Sweep}

We perform a detailed parameter sweep for the inflation response coefficient ($\psi_1$), ranging from 1.1 to 3.0. Figure \ref{fig:sens_sweep} illustrates the results. As the central bank becomes more aggressive, the volatility of inflation is reduced, but at the cost of more significant initial fluctuations in the nominal interest rate and a sharper, more front-loaded output contraction.

\begin{figure}[htbp]
    \centering
    \includegraphics[width=1.0\textwidth]{../../outputs/plots/sensitivity_psi1_sweep.png}
    \caption{Sensitivity Sweep: Varying $\psi_1$ from 1.1 (light) to 3.0 (dark)}
    \label{fig:sens_sweep}
\end{figure}

\section{Conclusion}
The M1002 model provides a robust framework for understanding macroeconomic fluctuations. Our experiments underscore that the effectiveness of monetary policy and the inherent rigidities in price setting are critical determinants of economic stability.

\end{document}
