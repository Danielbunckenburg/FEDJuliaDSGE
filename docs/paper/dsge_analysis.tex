\documentclass[11pt,a4paper]{article}
\usepackage[utf8]{inputenc}
\usepackage{amsmath, amssymb, amsthm}
\usepackage{graphicx}
\usepackage{booktabs}
\usepackage{hyperref}
\usepackage{geometry}
\geometry{margin=1in}

\title{Analysis of the US Macroeconomic Dynamics: \\ A DSGE Model Perspective}
\author{Antigravity \& Daniel}
\date{\today}

\begin{document}

\maketitle

\begin{abstract}
This paper investigates the dynamic properties of a medium-scale New Keynesian DSGE model (M1002). We perform several experiments, including Impulse Response Function (IRF) analysis for monetary and technology shocks, and sensitivity tests on key structural parameters. Our results highlight the importance of the Taylor rule's aggressiveness and price stickiness in shaping the transmission of shocks to the real economy.
\end{abstract}

\section{Introduction}
Modern macroeconomics relies heavily on Dynamic Stochastic General Equilibrium (DSGE) models for policy analysis and forecasting. This study focuses on the M1002 model, which incorporates several frictions, including price and wage stickiness, habit formation, and financial frictions.

\section{The Model}
The model is a medium-scale New Keynesian framework with several nominal and real rigidities. Key structural parameters are summarized in Table \ref{tab:params}.

\begin{center}
\begin{table}
\caption{Model Parameters}
\label{tab:params}
\begin{tabular}{llll}
\toprule
Name & Symbol & Value & Description \\
\midrule
alp & \alpha & 0.1596 &  \\
zeta_p & \zeta_p & 0.8940 &  \\
iota_p & \iota_p & 0.1865 &  \\
h & h & 0.5347 &  \\
ppsi & \psi & 0.6862 &  \\
psi1 & \psi_1 & 1.3679 &  \\
psi2 & \psi_2 & 0.0388 &  \\
psi3 & \psi_3 & 0.2464 &  \\
rho & \rho_R & 0.7126 &  \\
pi_star & \pi_* & 1.0050 &  \\
\bottomrule
\end{tabular}
\end{table}

\end{center}

\subsection{Key Equilibrium Equations}
The dynamics of the economy are governed by a system of linearized stochastic difference equations. Some of the most critical equations include:

\begin{itemize}
    \item \textbf{Consumption Euler Equation:}
    Deriving from the household's optimization problem with habit formation $h$:
    \[ c_t = \frac{h/e^{z^*}}{1+h/e^{z^*}} c_{t-1} + \frac{1}{1+h/e^{z^*}} E_t[c_{t+1}] - \frac{1-h/e^{z^*}}{\sigma_c(1+h/e^{z^*})} (R_t - E_t[\pi_{t+1}] + E_t[z_{t+1}]) \]
    where $z_t$ is the technology growth rate and $\sigma_c$ governs the intertemporal elasticity of substitution.

    \item \textbf{New Keynesian Phillips Curve (NKPC):}
    Inflation $\pi_t$ depends on future expected inflation, past inflation (via indexation $\iota_p$), and marginal cost $mc_t$:
    \[ \pi_t = \frac{\iota_p}{1+\beta \iota_p} \pi_{t-1} + \frac{\beta}{1+\beta \iota_p} E_t[\pi_{t+1}] + \kappa mc_t + \lambda_{f,t} \]
    where $\kappa$ is a function of the Calvo price stickiness parameter $\zeta_p$.

    \item \textbf{Taylor Rule:} 
    The central bank follows a smoothed interest rate rule:
    \[ R_t = \rho R_{t-1} + (1-\rho)[\psi_1 (\pi_t - \pi^*) + \psi_2 (y_t - y_t^f)] + \psi_3 \Delta y_t + r_{m,t} \]
    where $y_t^f$ is the level of output that would prevail under flexible prices.

    \item \textbf{Resource Constraint:}
    Output is split between consumption, investment, and government spending:
    \[ y_t = \frac{c^*}{y^*} c_t + \frac{i^*}{y^*} i_t + g_t + \dots \]
\end{itemize}

\subsection{Data and Identification}
The model is designed to be estimated using live macroeconomic data from the Federal Reserve Economic Data (FRED) database. We utilize 13 key observables:
\begin{enumerate}
    \item \textbf{Output Growth}: Real GDP per capita growth (GDP, CNP16OV).
    \item \textbf{Inflation}: GDP Deflator (GDPDEF) and Core PCE (PCEPILFE).
    \item \textbf{Interest Rates}: Effective Federal Funds Rate (DFF).
    \item \textbf{Labor Market}: Average weekly hours and compensation (AWHNONAG, COMPNFB).
    \item \textbf{Financial Spreads}: BAA-10Y Treasury spread.
\end{enumerate}

Raw data is transformed from levels to log-deviations or growth rates to match the model's stationary states. 

\subsection{Calibration and Methodology}
The model parameters are calibrated or estimated using a Bayesian approach. Parameters like the depreciation rate ($\delta = 0.025$) and the steady-state capital share ($\alpha = 0.3$) are often fixed to standard values, while others like price stickiness ($\zeta_p$) and Taylor rule coefficients ($\psi_i$) are estimated by maximizing the posterior density:
\[ \mathcal{P}(\theta | Y) \propto \mathcal{L}(Y | \theta) \pi(\theta) \]
where $\mathcal{L}(Y | \theta)$ is the likelihood computed via the Kalman Filter and $\pi(\theta)$ represents the prior distributions.

\section{Experiments and Results}

\subsection{Baseline Impulse Responses}
We analyze how the economy reacts to exogenous shocks over a 40-quarter horizon. Figure \ref{fig:irf_monetary} shows the impact of a contractionary monetary policy shock. 

\begin{figure}[htbp]
    \centering
    \includegraphics[width=0.8\textwidth]{../../outputs/plots/irf_rm_sh.png}
    \caption{IRFs - Monetary Policy Shock (40 Quarters)}
    \label{fig:irf_monetary}
\end{figure}

Figure \ref{fig:irf_tech} displays the response to a neutral technology shock. A boost in productivity increases output while putting downward pressure on inflation.

\begin{figure}[htbp]
    \centering
    \includegraphics[width=0.8\textwidth]{../../outputs/plots/irf_ztil_sh.png}
    \caption{IRFs - Technology Shock (40 Quarters)}
    \label{fig:irf_tech}
\end{figure}

\subsection{Sensitivity Analysis: Taylor Rule Sweep}

We perform a detailed parameter sweep for the inflation response coefficient ($\psi_1$), ranging from 1.1 to 3.0. Figure \ref{fig:sens_sweep} illustrates the results. As the central bank becomes more aggressive, the volatility of inflation is reduced, but at the cost of more significant initial fluctuations in the nominal interest rate and a sharper, more front-loaded output contraction.

\begin{figure}[htbp]
    \centering
    \includegraphics[width=1.0\textwidth]{../../outputs/plots/sensitivity_psi1_sweep.png}
    \caption{Sensitivity Sweep: Varying $\psi_1$ from 1.1 (light) to 3.0 (dark)}
    \label{fig:sens_sweep}
\end{figure}

\section{Conclusion}
The M1002 model provides a robust framework for understanding macroeconomic fluctuations. Our experiments underscore that the effectiveness of monetary policy and the inherent rigidities in price setting are critical determinants of economic stability.

\end{document}
