\documentclass[11pt,a4paper]{article}
\usepackage[utf8]{inputenc}
\usepackage{amsmath, amssymb, amsthm}
\usepackage{graphicx}
\usepackage{booktabs}
\usepackage{hyperref}
\usepackage{geometry}
\geometry{margin=1in}

\title{Analysis of the US Macroeconomic Dynamics: \\ A DSGE Model Perspective}
\author{Antigravity \& Daniel}
\date{\today}

\begin{document}

\maketitle

\begin{abstract}
This paper investigates the dynamic properties of a medium-scale New Keynesian DSGE model (M1002). We perform several experiments, including Impulse Response Function (IRF) analysis for monetary and technology shocks, and sensitivity tests on key structural parameters. Our results highlight the importance of the Taylor rule's aggressiveness and price stickiness in shaping the transmission of shocks to the real economy.
\end{abstract}

\section{Introduction}
Modern macroeconomics relies heavily on Dynamic Stochastic General Equilibrium (DSGE) models for policy analysis and forecasting. This study focuses on the M1002 model, which incorporates several frictions, including price and wage stickiness, habit formation, and financial frictions.

\section{The Model}
The model is a medium-scale New Keynesian framework. Key structural parameters are summarized in Table \ref{tab:params}.

\begin{center}
\begin{table}
\caption{Model Parameters}
\label{tab:params}
\begin{tabular}{llll}
\toprule
Name & Symbol & Value & Description \\
\midrule
alp & \alpha & 0.1596 &  \\
zeta_p & \zeta_p & 0.8940 &  \\
iota_p & \iota_p & 0.1865 &  \\
h & h & 0.5347 &  \\
ppsi & \psi & 0.6862 &  \\
psi1 & \psi_1 & 1.3679 &  \\
psi2 & \psi_2 & 0.0388 &  \\
psi3 & \psi_3 & 0.2464 &  \\
rho & \rho_R & 0.7126 &  \\
pi_star & \pi_* & 1.0050 &  \\
\bottomrule
\end{tabular}
\end{table}

\end{center}

Key components include:
\begin{itemize}
    \item \textbf{Taylor Rule:} The central bank adjusts the nominal interest rate $R_t$ in response to inflation $\pi_t$ and output growth $y_t$.
    \[ R_t = \rho_R R_{t-1} + (1-\rho_R)[\psi_1 (\pi_t - \pi_*) + \psi_2 (y_t - y_f,t)] + \dots \]
    \item \textbf{Phillips Curve:} Prices are set by firms in a monopolistically competitive market with Calvo-style indexation ($\zeta_p$).
\end{itemize}

\section{Experiments and Results}

\subsection{Baseline Impulse Responses}
We analyze how the economy reacts to exogenous shocks over a 40-quarter horizon. Figure \ref{fig:irf_monetary} shows the impact of a contractionary monetary policy shock. 

\begin{figure}[htbp]
    \centering
    \includegraphics[width=0.8\textwidth]{../../outputs/plots/irf_rm_sh.png}
    \caption{IRFs - Monetary Policy Shock (40 Quarters)}
    \label{fig:irf_monetary}
\end{figure}

Figure \ref{fig:irf_tech} displays the response to a neutral technology shock. A boost in productivity increases output while putting downward pressure on inflation.

\begin{figure}[htbp]
    \centering
    \includegraphics[width=0.8\textwidth]{../../outputs/plots/irf_ztil_sh.png}
    \caption{IRFs - Technology Shock (40 Quarters)}
    \label{fig:irf_tech}
\end{figure}

\subsection{Sensitivity Analysis: Taylor Rule Sweep}

We perform a detailed parameter sweep for the inflation response coefficient ($\psi_1$), ranging from 1.1 to 3.0. Figure \ref{fig:sens_sweep} illustrates the results. As the central bank becomes more aggressive, the volatility of inflation is reduced, but at the cost of more significant initial fluctuations in the nominal interest rate and a sharper, more front-loaded output contraction.

\begin{figure}[htbp]
    \centering
    \includegraphics[width=1.0\textwidth]{../../outputs/plots/sensitivity_psi1_sweep.png}
    \caption{Sensitivity Sweep: Varying $\psi_1$ from 1.1 (light) to 3.0 (dark)}
    \label{fig:sens_sweep}
\end{figure}

\section{Conclusion}
The M1002 model provides a robust framework for understanding macroeconomic fluctuations. Our experiments underscore that the effectiveness of monetary policy and the inherent rigidities in price setting are critical determinants of economic stability.

\end{document}
